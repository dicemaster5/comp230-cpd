% Please do not change the document class
\documentclass{scrartcl}

% Please do not change these packages
\usepackage[hidelinks]{hyperref}
\usepackage[none]{hyphenat}
\usepackage{setspace}
\doublespace

% You may add additional packages here
\usepackage{amsmath}

% Please include a clear, concise, and descriptive title
\title{Continuing Professional Development - Semester One}

% Please do not change the subtitle
\subtitle{COMP130 - CPD Report}

% Please put your student number in the author field
\author{1703086}

\begin{document}

\maketitle

\section{Introduction}
My current long-term career goal is still to become a lead game developer in a Indie studio in which I can have a lot of creative freedom, and to be able to develop my own personal game ideas into actual games with a team. My current academic development goals are to overall become a better programmer and games developer, to learn as much about game development as possible and to work more effectively as a team member and or team leader.
\\
The 5 key skills that I have identified from this first semester that need further improvement are: Dealing with frustrating team members; Leadership skills; Dealing with procrastination; Better understanding Shaders, how they work and how to write them; Using programming patterns to my advantage more often.

%Write your introduction here. A brief introduction of about 100 words is recommended, which should state your career goal and the five key skills that you wish to highlight from your weekly reports. When choosing which skills to focus on for this report, be specific. Avoid choosing broad skills that are clearly important for any student, such as \textit{time management} or \textit{communication}. Instead, make it more granular. Consider which specific aspects of these broad areas are a priority for you, personally, and what may have caused or exacerbated the challenge. Tutors are not assessing your knowledge of general study skills. Rather, they are assessing your ability to analyse and reflect on your own learning and personal development as an individual and towards becoming a computing professional.

\section{First Key Skill: Dealing with frustrating team members}
This semester I have had to deal with team members which I find difficult to work with and I have been finding myself easily agitated around these people. I need to try my best to not get frustrated around these people as this is just one of the things you get when working in teams with other people, disagreements happen and you just need to work them out and carry on.
\\
\\
Smart action: Over the next semester I'm going to learn to be less agitated around these people by changing my mindset around them, I'll try to better understand where they are coming from and what they are trying to communicate to the team even if I might disagree with them.
\\
Hopefully by the end of next semester I will have this well under control.

%Write about 200 words about. Remember, this is should be reflective and personal to you. Justify the relevance and importance of each of these skills with insight into your personal goals and personal circumstances. Assess your application of the skill throughout the semester and critically reflect on upon their impact it has had on your work and the challenges/obstacles. Acknowledge difficulties. Then, suggest how to overcome the challenge/obstacle in relation to a SMART action. When planning such actions, do not be too general. Consider SMART actions:
%specific measurable; achievable; relevant; and time-bound. Ensure that your proposed action for future development meets all five of these criteria.

\section{Second Key Skill: Leadership skills}
This semester I became the scrum master of my team. Which led to me having quite a few leadership responsibilities as other team members expected me to manage and deal with organizing everyone's agile scrum tasks and to set up scrum meetings and do daily stand ups. It's been going well so far but I know I could definitely improve on my skills and better manage team communication when doing stand ups and scrum meetings.
\\
\\
Smart action: My goal for next semester is to set a specific time to do a stand up with the team everyday and make sure It gets carried out. I'm also going to look into reading "peopleware:Productive Projects and Teams" as I have heard that it's a really good book for learning to manage teams and projects relating to software engineering
I will see if the library has this book available and if not I will look into buying it.
\\
By the end of next semester I want to have read the book and managed to do a daily stand up every day with my team. Hopefully this will show in the quality of our game when it is finished at the end of next semester.

\section{Third Key Skill: Dealing with procrastination}
Over the semester I have found myself procrastinating a lot on tasks or projects that I didn't really enjoy doing. It's a terrible waste of time and I really need to reduce and stop my procrastinating habits.
\\
\\
Smart action: During the second semester I'm going to stop my procrastinating habits by replacing them with better habits, such as working on a different task instead of the one I'm having difficult with, taking quick small breaks to walk around if I catch myself procrastinating and getting rid of distractions around me.
\\
I'm not to sure on how I'm going to measure my progress on becoming better at dealing with procrastination but by the end of the second semester I want to have at least cut my procrastination time down by 75\% to increase my overall productivity and save time.


\section{Fourth Key Skill: Better understanding Shaders, how they work and how to write them}
Graphics programming and shader development is a technical subject I find very interesting and I am very keen in learning about.
\\
This semester I learnt a lot about graphics programming and shaders in COMP220 when creating the graphics demos in OpenGL, but I still feel like I don't quite fully understand how they work yet and would like to better understand how to make them so I can then apply them to the projects I am currently working on.
\\
\\
Smart action: Next semester I'm going to read through Thebookofshaders.com to learn more about the concepts and uses of Graphics programming and shaders. And I will set a goal of creating 5 different shaders that could be applied to my team game in Unity using HLSL by following the many different tutorials that are available online.
\\
So by the end of next semseter I want to have created at least 5 different shaders to be potentially used for my team game and to have a much better understanding of how they work so I can carry on developing shaders for future games.
\section{Fifth Key Skill: Using programming patterns to my advantage more often}
During the summer before this semester I read some of the Game Programming patterns book by Robert Nystrom to get more familiar with programming patterns, and it was definitely helpful in getting a better understanding of programming patterns, but being fully aware of when I need to implement these patterns in my own code is still something I'm still having trouble with.
\\
\\
Smart action: Next semester I will create a reference sheet of a list of programming patterns I think will be of the most use to me and have simplified descriptions for each pattern to allow quick and easy referencing, and I will use the reference sheet whenever I'm programming to have it to give myself ideas on where I could implement certain programming patterns.
\\
By the end of the semester I want to have a nice Programming patterns reference sheet to use and to have used it for implementing at least 6 different programming patterns over the semester in order to acquire a better understanding on how to use programming patterns.

%During this semester I learnt a lot about design and programming patterns and other computing concepts that I could of tried to implement in my work and work flow that probably could of helped quite a lot but didn't feel comfortable enough to use effectively to want to actually try out.
%This is a shame as I think I would of learnt a lot more from actually using some of these concepts. It might of even helped me to be more efficient and less wasteful when it came to creating new game design concepts with my designer in my team, we had to scrap a lot of good ideas and work on things that never got implemented in the end because it wasn't any fun. 
%Over the summer my goal is to look back at these computing concepts and try to apply them while working on my summer projects, I'm also going to read into Game Programming patterns by Robert Nystrom.

\section{Conclusion}
The 5 SMART goals that I have set myself will help me to become a better team member and hopefully also a better team leader, it will help me be more efficient in programming and learn to better understand and use game programming design concepts making me a better programmer with more desirable traits so I can one day land a games programmer job.


%Write your conclusion here. Though the conclusion should be brief, no more than 100 words, it should do more than merely summarise the report. Focus on the five SMART actions that you intend to take in order to overcome any challenges and/or obstacles. Contextualise how this will help you towards your intended career goal and how this may improve your project for the next semester.

\bibliographystyle{ieeetran}
\bibliography{references}

\end{document}
